\documentclass[12pt]{article}

\usepackage{xltxtra}
\defaultfontfeatures{Ligatures=TeX}
\usepackage{fontspec}
\usepackage{xcolor}
\usepackage{fullpage}
\usepackage[color]{coqdoc}
\usepackage{amsmath,amssymb}
\usepackage{url}
\usepackage{makeidx,hyperref}
\usepackage{minted}
\usepackage{changepage}

\newcommand{\naive}[0]{na\"ive}
\newcommand{\pipes}{\texttt{pipes}}

\title{Formalizing Pipes in Coq}
\author{John Wiegley \and Gabriel Gonzalez}

\makeindex

\begin{document}

\setlength{\parskip}{0.5ex}
\maketitle

\thispagestyle{empty}
\mbox{}\vfill
\begin{center}

Copyright John Wiegley 2015.


This work is licensed under a
Creative Commons Attribution-Noncommercial-No Derivative Works 3.0
Unported License.
The license text is available at:

\end{center}

\begin{center} \url{http://creativecommons.org/licenses/by-nc-nd/3.0/} \end{center}

\phantomsection
\tableofcontents

\section{Introduction}

%\include{Intro.v}

\section{Pipes.Internal}

The first step toward formalization is to transport the core \pipes{}
\verb|Proxy| type into Coq.  This represents a difficulty since the Haskell
type is not strictly positive:

\begin{adjustwidth}{2em}{}
\begin{minted}{haskell}
data Proxy a' a b' b m r
    = Request a' (a  -> Proxy a' a b' b m r )
    | Respond b  (b' -> Proxy a' a b' b m r )
    | M          (m    (Proxy a' a b' b m r))
    | Pure    r
\end{minted}
\end{adjustwidth}

To work around this, the contravariant Yoneda lemma is used, made applicable
to covariant functors by exchanging the universally quantified function it
represents to an existentially quantified pairing:

\begin{adjustwidth}{2em}{}
\usemintedstyle{friendly}
\begin{minted}{coq}
Inductive Proxy (a' a b' b : Type) (m : Type -> Type) (r : Type) :=
  | Request of a' & (a  -> Proxy a' a b' b m r)
  | Respond of b  & (b' -> Proxy a' a b' b m r)
  | M x     of (x -> Proxy a' a b' b m r) & m x
  | Pure    of r.
\end{minted}
\end{adjustwidth}

\section{Pipes.Core}

%\include{src/Pipes/Core.v}

\section{Pipes}

%\include{src/Pipes.v}

\clearpage
\addcontentsline{toc}{section}{Bibliography}
\bibliographystyle{plain}
\bibliography{coq-pipes}

\clearpage
\addcontentsline{toc}{section}{Index}
\printindex

\end{document}
